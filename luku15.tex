\chapter{NP-ongelmat}

Olemme tässä kirjassa tutustuneet menetelmiin, joiden avulla voimme
ratkaista tehokkaasti ongelmia.
Kuitenkin on myös monia ongelmia, joiden ratkaisemiseen ei tällä
hetkellä tunneta mitään tehokasta algoritmia.
Jos vastaamme tulee tällainen ongelma, hyvät neuvot ovat kalliit.

Vaikeiden ongelmien yhteydessä vastaan tulee usein
kirjainyhdistelmä NP.
Erityisen tunnettu on P vs. NP -ongelma,
jonka ratkaisijalle on luvattu miljoonan dollarin potti.
Hankalalta tuntuvasta ongelmasta saatetaan mainita, että
se on NP-täydellinen tai NP-vaikea.
Nyt on aika selvittää, mitä nämä käsitteet oikeastaan tarkoittavat.

\section{Luokat P ja NP}

Keskitymme tässä luvussa \emph{päätösongelmiin}, joissa algoritmin tulee
antaa vastaus ''kyllä'' tai ''ei''.
Esimerkiksi ongelma 
''onko verkossa polkua solmusta $a$ solmuun $b$?'' on päätösongelma.
Tulemme huomaamaan, että voimme muotoilla monenlaisia ongelmia
päätösongelmina eikä tämä rajoita juurikaan, mitä ongelmia voimme tarkastella.

\subsubsection{Luokka P}

Luokka P sisältää päätösongelmat, joiden ratkaisemiseen on
olemassa \emph{polynominen} algoritmi eli algoritmi, jonka aikavaativuus
on enintään $O(n^k)$, missä $k$ on vakio.
Lähes kaikki tässä kirjassa esitetyt algoritmit
ovat toimineet polyonmisessa ajassa.
Tuttuja polynomisia aikavaativuuksia ovat esimerkiksi
$O(1)$, $O(\log n)$ $O(n)$, $O(n \log n)$, $O(n^2)$ ja $O(n^3)$.

Esimerkiksi ongelma ''onko verkossa polkua solmusta $a$ solmuun $b$?''
kuuluu luokkaan P, koska voimme ratkaista sen monellakin tavalla
polynomisessa ajassa.
Voimme vaikkapa aloittaa syvyyshaun solmusta $a$ ja tarkastaa,
pääsemmekö solmuun $b$.
Tuloksena on algoritmi, jolla on polynominen aikavaativuus
$O(n+m)$, joten ongelma kuuluu luokkaan P.

Luokan P tarkoituksena on kuvata ongelmia, jotka voimme
ratkaista jossain mielessä \emph{tehokkaasti}.
Tässä tehokkuuden määritelmä on varsin karkea:
pidämme algoritmia tehokkaana, jos sillä on mikä tahansa
polynominen aikavaativuus.
Onko $O(n^{100})$-aikainen algoritmi siis tehokas?
Ei, mutta käytännössä vakio $k$ on yleensä pieni ja
on osoittautunut, että polynominen aikavaativuus 
on toimiva tehokkuuden mittari.


\subsubsection{Luokka NP}

Luokka NP sisältää päätösongelmat, joissa jokaisessa
''kyllä''-tapauksessa on olemassa
\emph{todiste}, jonka avulla voimme
\emph{tarkastaa} polynomisessa ajassa, että vastaus on oikein.
Todiste on merkkijono, jonka koko on polynominen
suhteessa syötteeseen,
ja se antaa meille lisätietoa siitä,
minkä takia ''kyllä''-vastaus pitää paikkansa syötteelle.

Esimerkki luokkaan NP kuuluvasta ongelmasta on
''onko verkossa polkua solmusta $a$ solmuun $b$,
joka kulkee tasan kerran jokaisen verkon solmun kautta?''.
Tämän ongelman ratkaisemiseen ei tunneta tehokasta algoritmia,
mutta jokaisessa ''kyllä''-tapauksessa on olemassa todiste:
halutunlainen polku solmusta $a$ solmuun $b$.
Voimme tarkastaa helposti polynomisessa ajassa,
että todisteen kuvaamalla polulla on vaaditut ominaisuudet.

Jos vastaus syötteeseen on ''ei'', tähän ei tarvitse
liittyä tehokkaasti tarkastettavaa todistetta.
Usein olisikin hankalaa antaa todiste siitä, että jotain
asiaa \emph{ei} ole olemassa.
Äskeisessä esimerkissä meidän oli helppoa todistaa
polun olemassaolo, koska voimme vain näyttää kyseisen polun,
mutta meillä ei ole vastaavaa keinoa todistaa, että polkua ei ole olemassa.

Huomaa, että kaikki luokan P ongelmat kuuluvat myös
luokkaan NP. Tämä johtuu siitä, että luokan P ongelmissa
voimme tarkastaa ''kyllä''-vastauksen
\emph{tyhjän} todisteen avulla: voimme saman tien ratkaista
koko ongelman alusta alkaen polynomisessa ajassa.

\subsubsection{P vs. NP}

Äkkiseltään tuntuu selvältä, että luokassa NP täytyy olla
enemmän ongelmia kuin luokassa P.
Luokassa NP meidän riittää vain pystyä tarkastamaan ''kyllä''-vastauksen
todiste, mikä tuntuu helpommalta kuin muodostaa ongelman ratkaisu tyhjästä.
Monet uskovatkin, että luokka NP on suurempi kuin luokka P --
mutta kukaan ei ole onnistunut todistamaan asiaa.

Tietojenkäsittelytieteen merkittävä avoin ongelma onkin,
päteekö $P=NP$ vai $P \neq NP$.
Monet tutkijat ovat tarttuneet haasteeseen
70-luvulta lähtien, mutta kaikki ovat epäonnistuneet.
Ongelman ratkaisija saisi maineen ja kunnian lisäksi
myös tuntuvan rahallisen korvauksen, koska
Clay-instituutti on luvannut miljoonan dollarin palkinnon
sille, joka todistaa, että $P=NP$ tai $P \neq NP$.
Voi olla kuitenkin, että tämä on yksi \emph{vaikeimmista}
tavoista ansaita miljoona dollaria.

Jos pätee $P \neq NP$, kuten uskotaan,
vaikeutena on keksiä keino todistaa, että jotakin
luokan NP ongelmaa on mahdotonta ratkaista
polynomisessa ajassa.
Tämän todistaminen on vaikeaa, koska meidän pitää näyttää,
että tehokasta algoritmia ei ole olemassa,
vaikka laatisimme algoritmin miten tahansa.
Vaikka moni on koettanut tuloksetta ratkoa
tunnettuja NP-ongelmia, kysymys saattaa silti olla siitä,
että tehokas algoritmi olisi olemassa mutta kukaan ei
vain ole vielä löytänyt sitä.

\section{NP-täydellisyys}

Sanomme, että ongelma on \emph{NP-täydellinen},
jos se kuuluu luokkaan NP ja mikä tahansa luokan NP
ongelma voidaan \emph{palauttaa} siihen polynomisessa ajassa.
NP-täydelliset ongelmat ovat luokan NP vaikeimpia ongelmia:
jos voisimme ratkaista jonkin NP-täydellisen ongelman tehokkaasti,
voisimme ratkaista minkä tahansa luokan NP ongelman tehokkaasti.

Kiinnostava ilmiö on, että lähes kaikki tunnetut luokan NP
ongelmat joko kuuluvat myös luokkaan P tai ovat
NP-täydellisiä.
Nykyään tunnetaankin tuhansia erilaisia NP-täydellisiä ongelmia.
Jos keksisimme mihin tahansa niistä polynomisessa ajassa toimivan
ratkaisun, olisimme samalla onnistuneet todistamaan, että $P=NP$.

\subsection{SAT-ongelma}

Ensimmäinen löydetty NP-täydellinen ongelma oli
SAT-ongelma, jossa annettuna on konjunktiivisessa
normaalimuodossa oleva looginen kaava ja haluamme
selvittää, voimmeko valita muuttujien arvot niin,
että kaava on tosi.
Konjunktiivinen normaalimuoto tarkoittaa,
että kaava koostuu osista, jotka on yhdistetty
ja-operaatiolla ($\land$), ja jokainen osa puolestaan koostuu
muuttujista ja niiden negaatioista, jotka on yhdistetty
tai-operaatiolla ($\lor$).

Esimerkiksi kaava
\[(\neg x_1 \lor x_3) \land (x_1 \lor x_2 \lor x_3) \land (\neg x_2 \lor \neg x_3)\]
on mahdollista saada todeksi, koska voimme esimerkiksi asettaa
muuttujat $x_1$ ja $x_2$ epätosiksi ja muuttujan $x_3$ todeksi.
Vastaavasti kaava
\[(x_1 \lor x_2) \land (\neg x_1 \lor x_2) \land (x_1 \lor \neg x_2) \land (\neg x_1 \lor \neg x_2) \]
ei ole tosi, vaikka valitsisimme muuttujien arvot miten tahansa.

Kun haluamme osoittaa, että SAT on NP-täydellinen ongelma,
meidän täytyy näyttää, että se kuuluu luokkaan NP ja mikä
tahansa luokan NP ongelma voidaan palauttaa siihen.
Luokkaan NP kuuluminen on helppoa näyttää:
''kyllä''-tapauksessa todiste on jokaiselle muuttujalle valittu arvo.
Huomattavasti vaikeampaa on osoittaa, että \emph{jokainen} luokan
NP ongelma voidaan palauttaa SAT-ongelmaan polynomisessa ajassa.

Tässä kirjassa emme käsittele todistusta yksityiskohtaisesti,
mutta voimme kuitenkin kuvailla sen perusideaa.
Tarkastellaan tiettyä luokan NP ongelmaa,
joka meidän täytyy pystyä palauttamaan SAT-ongelmaan.
Koska ongelma voi olla mikä tahansa luokkaan NP kuuluva, tiedämme siitä vain,
että on olemassa algoritmi, joka tarkastaa
polynomisessa ajassa ''kyllä''-tapauksen todisteen.
Tämä vastaa sitä, että on olemassa epädeterministinen
Turingin kone, joka \emph{arvaa} ensin todisteen sisällön ja
tarkastaa sitten, että todiste on kelvollinen.
Ideana on muodostaa looginen kaava, joka luonnehtii tällaisen
Turing-koneen laskentaa ''kyllä''-tapauksessa:
koneen tulee käsitellä syöte jollakin tavalla,
muodostaa todiste ja tarkastaa se.
Kaavan tarkka sisältö riippuu siitä, miten kone on rakentunut,
mutta voimme aina muodostaa kaavan, kunhan tiedämme koneen rakenteen
ja annetun syötteen.
Tuloksena on looginen kaava, jonka voi saada todeksi tarkalleen silloin,
kun vastaus syötteeseen on ''kyllä'',
joten olemme palauttaneet ongelman SAT-ongelmaan.

\subsection{Ongelmien palautukset}

Kun meillä on tiedossa yksi NP-täydellinen ongelma,
voimme osoittaa muita ongelmia NP-täydellisiksi palautusten avulla.
Ideana on, että jos ongelma $A$ on NP-täydellinen ja
voimme palauttaa sen polynomisessa ajassa ongelmaksi $B$,
myös ongelma $B$ on NP-täydellinen.

Ongelman $A$ palauttaminen ongelmaksi $B$ tarkoittaa,
että näytämme yleisen tavan muuntaa ongelman $A$ syöte
ongelman $B$ syötteeksi.
Palautuksen täytyy säilyttää syötteen vastaus
(päätösongelmassa ''kyllä'' tai ''ei'') ennallaan.
Jos pystymme tekemään palautuksen, voimme keskittyä ratkomaan ongelman
$A$ sijasta ongelmaa $B$.

Näytämme seuraavaksi, miten voimme palauttaa SAT-ongelman
3SAT-ongelmaksi ja edelleen 3SAT-ongelman CLIQUE-ongelmaksi.
Tämä osoittaa, että sekä 3SAT että CLIQUE
ovat NP-täydellisiä ongelmia.

\subsubsection{Palautus SAT $\rightarrow$ 3SAT}

3SAT-ongelma on SAT-ongelman erikoistapaus, jossa jokaisessa
$\land$-merkeillä yhdistetyssä lausekkessa on tarkalleen kolme muuttujaa.
Haluamme näyttää, että voimme muuttaa minkä tahansa
SAT-ongelman syötteen 3SAT-ongelman syötteeksi,
jonka totuusarvo on sama.

Ideana on muokata jokaista lauseketta niin, 
että tuloksena on yksi tai useampia kolmen muuttujan lausekkeita.
Merkitään $k$ lausekkeen muuttujien määrää.
Jos $k=1$ tai $k=2$, toistamme viimeistä muuttujaa uudestaan,
jotta saamme aikaan kolme muuttujaa.
Esimerkiksi jos lauseke on $(x_1)$, muutamme sen muotoon
$(x_1 \lor x_1 \lor x_1)$, ja jos lauseke on $(x_1 \lor x_2)$, muutamme
sen muotoon $(x_1 \lor x_2 \lor x_2)$.

Jos $k=3$, meidän ei tarvitse tehdä mitään. Jos sitten $k>3$,
jaamme lausekkeen osiin, jotka ketjutetaan apumuuttujien avulla.
Ketjun jokaisessa kohdassa vasemman lausekkeen viimeinen
muuttuja on $a_i$ ja oikean lausekkeen ensimmäinen muuttuja on $\neg a_i$.
Tämä takaa, että ainakin yksi alkuperäinen muuttuja saa oikean arvon.
Esimerkiksi jos lauseke on $(x_1 \lor x_2 \lor x_3 \lor x_4 \lor x_5)$,
muutamme sen kolmeksi lausekkeeksi $(x_1 \lor x_2 \lor a_1)$,
$(\neg a_1 \lor x_3 \lor a_2)$ ja $(\neg a_2 \lor x_4 \lor x_5)$.

Tämä palautus osoittaa, että 3SAT on NP-täydellinen ongelma --
eli SAT-ongelman oleellinen vaikeus syntyy siitä, että lausekkeissa
voi olla kolme muuttujaa\footnote{Entä ongelma 2SAT, jossa jokaisessa lausekkeessa
on kaksi muuttujaa? Tämä \emph{ei} ole NP-täydellinen ongelma,
vaan kuuluu luokkaan P.}.
Palautuksen hyötynä on toisaalta myös, että kolmen muuttujan lausekkeita
on helpompaa käsitellä myöhemmissä todistuksissa
kuin vaihtelevan pituisia lausekkeita.


\subsubsection{Palautus 3SAT $\rightarrow$ CLIQUE}

\subsection{Esimerkkejä}

\subsection{NP-vaikeat ongelmat}

Sanomme, että ongelma on NP-vaikea, jos voimme palauttaa
NP-täydellisen ongelman siihen mutta ongelma ei välttämättä
kuulu luokkaan NP.
Jos ongelma on NP-vaikea, se on siis ainakin yhtä vaikea
kuin luokan NP vaikeimmat ongelmat.

Tähän mennessä olemme käsitelleet päätösongelmia,
mutta käytännössä moni ongelma \emph{ei} ole päätösongelma
vaan optimointiongelma: haluamme löytää jollakin
tavalla parhaan ratkaisun ongelmaan.
Vaikuttaa siis, että päätöson\-gelmien ulkopuolelle jää
monia kiinnostavia ongelmia.

Osoittautuu kuitenkin, että voimme ratkaista optimointiongelmia
muotoilemalla ne päätösongelman avulla.
Lisäksi jos meillä on tehokas ratkaisu päätösongelmaan,
pystymme sen avulla saamaan aikaan myös tehokkaan
ratkaisun optimointiongelmaan.
Ideana on muuttaa optimointiongelma päätös\-ongelmaksi niin,
että kysymme, onko olemassa ratkaisua, jonka hyvyys ylittää
tietyn rajan $x$. Tämän jälkeen voimme selvittää binäärihaun
avulla, mikä on suurin $x$:n arvo, jolle ratkaisu on olemassa.

Tarkastellaan esimerkkinä ongelmaa, jossa haluamme selvittää,
mikä on verkon suurin \emph{klikki} eli joukko solmuja,
jossa minkä tahansa kahden solmun välillä on kaari.
Voimme muotoilla tämän ongelman päätösongelmana
''onko verkossa klikkiä, jonka koko on vähintään $x$''
ja etsiä sitten suurimman $x$:n arvon binäärihaun avulla.