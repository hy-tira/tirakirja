\chapter{Suunnatut syklittömät verkot}

Verkkoalgoritmien suunnittelussa aiheuttavat
usein vaikeuksia verkossa olevat syklit,
ja monen ongelman ratkaiseminen on vaikeaa nimenomaan
sen takia, että meidän täytyy ottaa huomioon,
mitä tapahtuu sykleissä.
Tässä luvussa katsomme, miten asiat muuttuvat,
kun voimmekin olettaa, että käsitel\-tävänä on suunnattu verkko,
jossa \emph{ei ole} syklejä.

Kun verkko on suunnattu ja syklitön,
voimme muodostaa sille aina \emph{topologisen järjestyksen},
joka antaa meille luontevan järjestyksen käsitellä
verkon solmut niiden riippuvuuksien mukaisesti.
Tämän ansiosta voimme hyödyntää dynaamista ohjelmointia
verkon polkujen käsittelyssä.
Itse asiassa tulemme huomaamaan, että \emph{mikä tahansa}
dynaamisen ohjelmoinnin algoritmi voidaan nähdä
suunnatun syklittömän verkon käsittelynä.

Entä jos verkossa kuitenkin on syklejä?
Osoittautuu, että voimme silti esittää sen syvärakenteen
syklittömänä verkkona muodostamalla verkon
\emph{vahvasti yhtenäiset} komponentit.
Tämän ansiosta voimme tietyissä tilanteissa käsitellä
verkkoa mukavasti syklittömän verkon tavoin,
vaikka se ei olisikaan alun perin syklitön.

\begin{figure}[ht]
\center
\begin{center}
\begin{tikzpicture}[scale=0.6]
\small
\begin{scope}
\node[draw, circle] (1) at (0,-1) {$1$};
\node[draw, circle] (2) at (2,0) {$2$};
\node[draw, circle] (3) at (2,-2) {$3$};
\node[draw, circle] (4) at (4,0) {$4$};
\node[draw, circle] (5) at (4,-2) {$5$};
\path[draw,thick,->] (1) -- (2);
\path[draw,thick,->] (1) -- (3);
\path[draw,thick,->] (3) -- (2);
\path[draw,thick,->] (2) -- (4);
\path[draw,thick,->] (3) -- (4);
\path[draw,thick,->] (5) -- (4);
\end{scope}
\begin{scope}[xshift=7cm]
\node[draw, circle] (1) at (0,-1) {$1$};
\node[draw, circle] (3) at (2,-1) {$3$};
\node[draw, circle] (5) at (4,-1) {$5$};
\node[draw, circle] (2) at (6,-1) {$2$};
\node[draw, circle] (4) at (8,-1) {$4$};
\path[draw,thick,->] (1) edge [bend right] (2);
\path[draw,thick,->] (1) -- (3);
\path[draw,thick,->] (3) edge [bend left] (2);
\path[draw,thick,->] (2) -- (4);
\path[draw,thick,->] (3) edge [bend left] (4);
\path[draw,thick,->] (5) edge [bend right] (4);
\end{scope}
\end{tikzpicture}
\end{center}
\caption{Verkko ja yksi sen topologinen järjestys $[1,3,5,2,4]$.}
\label{fig:topjar}
\end{figure}

\section{Topologinen järjestys}

Topologinen järjestys on suunnatun verkon solmujen järjestys,
jossa pätee, että jos solmusta $a$ on kaari solmuun $b$,
niin solmu $a$ on ennen solmua $b$ järjestyksessä.
Topologinen järjestys voidaan esittää listana,
joka ilmaisee solmujen järjestyksen.
Kuvassa \ref{fig:topjar} on esimerkkinä verkko ja yksi sen topologinen
järjestys $[1,3,5,2,4]$.

Osoittautuu, että voimme muodostaa suunnatulle verkolle
topologisen järjestyksen tarkalleen silloin,
kun verkko on syklitön.
Tutustumme seuraavaksi tehokkaaseen algoritmiin,
joka muodostaa topologisen järjestyksen
tai toteaa, ettei järjestystä voi muodostaa
verkossa olevan syklin takia.

\subsection{Järjestyksen muodostaminen}

\begin{figure}
\center
\begin{center}
\begin{tikzpicture}[scale=0.6]
\scriptsize
\newcommand\verkko[6]{
\node[draw, circle, fill=#2] (1) at (0,-1) {$1$};
\node[draw, circle, fill=#3] (2) at (2,0) {$2$};
\node[draw, circle, fill=#4] (3) at (2,-2) {$3$};
\node[draw, circle, fill=#5] (4) at (4,0) {$4$};
\node[draw, circle, fill=#6] (5) at (4,-2) {$5$};
\path[draw,thick,->] (1) -- (2);
\path[draw,thick,->] (1) -- (3);
\path[draw,thick,->] (3) -- (2);
\path[draw,thick,->] (2) -- (4);
\path[draw,thick,->] (3) -- (4);
\path[draw,thick,->] (5) -- (4);
\node at (2,-3) {vaihe #1};
}
\begin{scope}
\verkko{1}{lightgray}{lightgray}{white}{white}{white}
\path[draw=red,thick,->,line width=2pt] (1) -- (2);
\end{scope}
\begin{scope}[xshift=6cm]
\verkko{2}{lightgray}{lightgray}{white}{lightgray}{white}
\path[draw=red,thick,->,line width=2pt] (1) -- (2);
\path[draw=red,thick,->,line width=2pt] (2) -- (4);
\end{scope}
\begin{scope}[xshift=12cm]
\verkko{3}{lightgray}{lightgray}{white}{gray}{white}
\path[draw=red,thick,->,line width=2pt] (1) -- (2);
\end{scope}
\begin{scope}[xshift=18cm]
\verkko{4}{lightgray}{gray}{white}{gray}{white}
\end{scope}
\begin{scope}[yshift=-5cm]
\verkko{5}{lightgray}{gray}{lightgray}{gray}{white}
\path[draw=red,thick,->,line width=2pt] (1) -- (3);
\end{scope}
\begin{scope}[yshift=-5cm,xshift=6cm]
\verkko{6}{lightgray}{gray}{lightgray}{gray}{white}
\path[draw=red,thick,->,line width=2pt] (1) -- (3);
\path[draw=red,thick,->,line width=2pt] (3) -- (2);
\end{scope}
\begin{scope}[yshift=-5cm,xshift=12cm]
\verkko{7}{lightgray}{gray}{lightgray}{gray}{white}
\path[draw=red,thick,->,line width=2pt] (1) -- (3);
\path[draw=red,thick,->,line width=2pt] (3) -- (4);
\end{scope}
\begin{scope}[yshift=-5cm,xshift=18cm]
\verkko{8}{lightgray}{gray}{gray}{gray}{white}
\path[draw=red,thick,->,line width=2pt] (1) -- (3);
\end{scope}
\begin{scope}[yshift=-10cm]
\verkko{9}{gray}{gray}{gray}{gray}{white}
\end{scope}
\begin{scope}[yshift=-10cm,xshift=6cm]
\verkko{10}{gray}{gray}{gray}{gray}{lightgray}
\path[draw=red,thick,->,line width=2pt] (5) -- (4);
\end{scope}
\begin{scope}[yshift=-10cm,xshift=12cm]
\verkko{11}{gray}{gray}{gray}{gray}{gray}
\end{scope}
\end{tikzpicture}
\end{center}
\caption{Esimerkki topologisen järjestyksen muodostamisesta.}
\label{fig:topesi}
\end{figure}

Voimme muodostaa topologisen järjestyksen suorittamalla
joukon syvyyshakuja, joissa jokaisella solmulla on kolme mahdollista tilaa:

\begin{itemize}
\item tila 0 (valkoinen): solmussa ei ole käyty
\item tila 1 (harmaa): solmun käsittely on kesken
\item tila 2 (musta): solmun käsittely on valmis
\end{itemize}

Algoritmin alussa jokainen solmu on valkoinen.
Käymme läpi kaikki verkon solmut ja aloitamme aina syvyyshaun
solmusta, jos se on valkoinen.
Aina kun saavumme uuteen solmuun, sen väri muuttuu
valkoisesta harmaaksi.
Sitten kun olemme käsitelleet kaikki solmusta lähtevät
kaaret, solmun väri muuttuu harmaasta mustaksi
ja lisäämme solmun listalle.
Tämä lista käänteisessä järjestyksessä on verkon
topologinen järjestys.
Kuitenkin jos saavumme jossain vaiheessa
toista kautta harmaaseen solmuun,
verkossa on sykli eikä topologista järjestystä ole olemassa.

\begin{figure}
\center
\begin{center}
\begin{tikzpicture}[scale=0.6]
\scriptsize
\newcommand\verkko[6]{
\node[draw, circle, fill=#2] (1) at (0,-1) {$1$};
\node[draw, circle, fill=#3] (2) at (2,0) {$2$};
\node[draw, circle, fill=#4] (3) at (2,-2) {$3$};
\node[draw, circle, fill=#5] (4) at (4,0) {$4$};
\node[draw, circle, fill=#6] (5) at (4,-2) {$5$};
\path[draw,thick,->] (1) -- (2);
\path[draw,thick,->] (1) -- (3);
\path[draw,thick,->] (3) -- (2);
\path[draw,thick,->] (2) -- (4);
\path[draw,thick,->] (4) -- (3);
\path[draw,thick,->] (5) -- (4);
\node at (2,-3) {vaihe #1};
}
\begin{scope}
\verkko{1}{lightgray}{lightgray}{white}{white}{white}
\path[draw=red,thick,->,line width=2pt] (1) -- (2);
\end{scope}
\begin{scope}[xshift=6cm]
\verkko{2}{lightgray}{lightgray}{white}{lightgray}{white}
\path[draw=red,thick,->,line width=2pt] (1) -- (2);
\path[draw=red,thick,->,line width=2pt] (2) -- (4);
\end{scope}
\begin{scope}[xshift=12cm]
\verkko{3}{lightgray}{lightgray}{lightgray}{lightgray}{white}
\path[draw=red,thick,->,line width=2pt] (1) -- (2);
\path[draw=red,thick,->,line width=2pt] (2) -- (4);
\path[draw=red,thick,->,line width=2pt] (4) -- (3);
\end{scope}
\begin{scope}[xshift=18cm]
\verkko{4}{lightgray}{lightgray}{lightgray}{lightgray}{white}
\path[draw=red,thick,->,line width=2pt] (1) -- (2);
\path[draw=red,thick,->,line width=2pt] (2) -- (4);
\path[draw=red,thick,->,line width=2pt] (4) -- (3);
\path[draw=red,thick,->,line width=2pt] (3) -- (2);
\end{scope}
\end{tikzpicture}
\end{center}
\caption{Topologista järjestystä ei voi muodostaa syklin takia.}
\label{fig:topsyk}
\end{figure}

Kuva \ref{fig:topesi} näyttää, kuinka algoritmi muodostaa topologisen
järjestyksen esimerkkiverkossamme.
Tässä tapauksessa suoritamme kaksi syvyyshakua,
joista ensimmäinen alkaa solmusta 1 ja toinen alkaa solmusta 5.
Algoritmin tuloksena on lista $[4,2,3,1,5]$,
joten käänteinen lista $[5,1,3,2,4]$ on verkon topologinen järjestys.
Huomaa, että tämä on eri järjestys kuin kuvassa \ref{fig:topjar} --
topologinen järjestys ei ole yksikäsitteinen ja voimme yleensä
muodostaa järjestyksen monella tavalla.
Kuva \ref{fig:topsyk} näyttää puolestaan tilanteen,
jossa topologista järjestystä ei voi muodostaa verkossa
olevan syklin takia.
Tässä verkossa on sykli $2 \rightarrow 4 \rightarrow 3 \rightarrow 2$,
jonka olemassaolon huomaamme siitä, että tulemme uudestaan
harmaaseen solmuun 2.

Miten voimme tietää, että algoritmimme on toimiva?
Tarkastellaan ensin tilannetta, jossa verkossa on sykli.
Jos algoritmi saapuu toista reittiä harmaaseen solmuun,
on selvää, että verkossa on sykli,
koska algoritmi on onnistunut pääsemään harmaasta solmusta
itseensä kulkemalla jotain polkua verkossa.
Toisaalta jos verkossa on sykli, algoritmi tulee
jossain vaiheessa ensimmäistä kertaa johonkin sykliin
kuuluvaan solmuun $x$. Tämän jälkeen se käy läpi solmusta
lähtevät kaaret ja aikanaan käy varmasti läpi kaikki syklin
solmut ja saapuu uudestaan solmuun $x$.
Niinpä algoritmi tunnistaa kaikissa tilanteissa verkossa olevan syklin.

Jos sitten verkossa ei ole sykliä, algoritmi lisää jokaisen
solmun listalle sen jälkeen, kun se on käsitellyt
kaikki solmusta lähtevät kaaret.
Jos siis verkossa on kaari $a \rightarrow b$,
solmu $b$ lisätään listalle ennen solmua $a$.
Lopuksi lista käännetään, jolloin solmu $a$
tulee ennen solmua $b$.
Tämän ansiosta jokaiselle kaarelle $a \rightarrow b$ pätee,
että solmu $a$ tulee järjestykseen ennen solmua $b$,
eli algoritmi muodostaa kelvollisen topologisen järjestyksen.

\subsection{Esimerkki: Kurssivalinnat}

Yliopiston kurssit ja niiden esitietovaatimukset voidaan esittää 
suunnattuna verkkona, jonka solmut ovat kursseja ja kaaret kuvaavat,
missä järjestyksessä kurssit tulisi suorittaa.

Kuvassa \ref{fig:kuresi} on esimerkkinä joitakin
tietojenkäsittely\-tieteen kursseja.
Tällaisen verkon topologinen järjestys antaa meille
yhden tavan suorittaa kurssit esitietovaatimusten mukaisesti.
Kuvassa \ref{fig:kurjar} näkyy esimerkkinä topologinen järjestys,
joka vastaa suoritusjärjestystä
OHPE, OHJA, TIPE, TITO, JYM, TIRA, LAMA.

\begin{figure}
\center
\begin{center}
\begin{tikzpicture}[scale=0.7]
\node[draw, rectangle] (1) at (0,0) {OHPE};
\node[draw, rectangle] (2) at (-4,-2) {OHJA};
\node[draw, rectangle] (3) at (4,-2) {TITO};
\node[draw, rectangle] (4) at (0,-2) {TIPE};
\node[draw, rectangle] (5) at (-8,-2) {JYM};
\node[draw, rectangle] (6) at (-4,-4) {TIRA};
\node[draw, rectangle] (7) at (-4,-6) {LAMA};
\path[draw,thick,->] (1) -- (2);
\path[draw,thick,->] (1) -- (3);
\path[draw,thick,->] (1) -- (4);
\path[draw,thick,->] (2) -- (4);
\path[draw,thick,->] (2) -- (6);
\path[draw,thick,->] (5) -- (6);
\path[draw,thick,->] (5) -- (7);
\path[draw,thick,->] (6) -- (7);
\end{tikzpicture}
\end{center}
\caption{Kurssien esitietovaatimukset verkkona.}
\label{fig:kuresi}
\end{figure}


\begin{figure}
\center
\begin{center}
\begin{tikzpicture}[scale=0.7]
\node[draw, rectangle] (1) at (0,0) {OHPE};
\node[draw, rectangle] (2) at (2.5,0) {OHJA};
\node[draw, rectangle] (4) at (5,0) {TIPE};
\node[draw, rectangle] (3) at (7.5,0) {TITO};
\node[draw, rectangle] (5) at (10,0) {JYM};
\node[draw, rectangle] (6) at (12.5,0) {TIRA};
\node[draw, rectangle] (7) at (15,0) {LAMA};
\path[draw,thick,->] (1) -- (2);
\path[draw,thick,->] (1) edge [bend left] (3);
\path[draw,thick,->] (1) edge [bend right] (4);
\path[draw,thick,->] (2) -- (4);
\path[draw,thick,->] (2) edge [bend left] (6);
\path[draw,thick,->] (5) -- (6);
\path[draw,thick,->] (5) edge [bend right] (7);
\path[draw,thick,->] (6) -- (7);
\end{tikzpicture}
\end{center}
\caption{Topologinen järjestys antaa kurssien suoritusjärjestyksen.}
\label{fig:kurjar}
\end{figure}

On selvää, että kurssien ja esitietovaatimusten muodostaman
verkon tulee olla syklitön, jotta kurssit voi suorittaa halutulla tavalla.
Jos verkossa on sykli, topologista järjestystä ei ole olemassa
eikä meillä ole mitään mahdollisuutta suorittaa kursseja
esitietovaatimusten mukaisesti.

\section{Dynaaminen ohjelmointi}

Kun tiedämme, että suunnattu verkko on syklitön,
voimme ratkaista helposti monia verkon polkuihin
liittyviä ongelmia \emph{dynaamisen ohjelmoinnin} avulla.
Tämä on mahdollista, koska topologinen järjestys antaa
meille selkeän järjestyksen, jossa voimme käsitellä solmut.

\subsection{Polkujen laskeminen}

Tarkastellaan esimerkkinä ongelmaa, jossa haluamme laskea,
montako polkua verkossa on solmusta $a$ solmuun $b$.
Esimerkiksi kuva \ref{fig:verpol} näyttää tilanteen,
jossa haluamme laskea polut solmusta $1$ solmuun $4$.
Tällaisia polkuja on kolme:
$1 \rightarrow 2 \rightarrow 4$,
$1 \rightarrow 3 \rightarrow 2 \rightarrow 4$ ja
$1 \rightarrow 3 \rightarrow 4$.

\begin{figure}
\center
\begin{center}
\begin{tikzpicture}[scale=0.6]
\small
\begin{scope}
\node[draw, circle] (1) at (0,-1) {$1$};
\node[draw, circle] (2) at (2,0) {$2$};
\node[draw, circle] (3) at (2,-2) {$3$};
\node[draw, circle] (4) at (4,0) {$4$};
\node[draw, circle] (5) at (4,-2) {$5$};
\path[draw,thick,->] (1) -- (2);
\path[draw,thick,->] (1) -- (3);
\path[draw,thick,->] (3) -- (2);
\path[draw,thick,->] (2) -- (4);
\path[draw,thick,->] (3) -- (4);
\path[draw,thick,->] (5) -- (4);
\path[draw=red,thick,->,line width=2pt] (1) -- (2);
\path[draw=red,thick,->,line width=2pt] (2) -- (4);
\end{scope}
\begin{scope}[xshift=6.5cm]
\node[draw, circle] (1) at (0,-1) {$1$};
\node[draw, circle] (2) at (2,0) {$2$};
\node[draw, circle] (3) at (2,-2) {$3$};
\node[draw, circle] (4) at (4,0) {$4$};
\node[draw, circle] (5) at (4,-2) {$5$};
\path[draw,thick,->] (1) -- (2);
\path[draw,thick,->] (1) -- (3);
\path[draw,thick,->] (3) -- (2);
\path[draw,thick,->] (2) -- (4);
\path[draw,thick,->] (3) -- (4);
\path[draw,thick,->] (5) -- (4);
\path[draw=red,thick,->,line width=2pt] (1) -- (3);
\path[draw=red,thick,->,line width=2pt] (3) -- (2);
\path[draw=red,thick,->,line width=2pt] (2) -- (4);
\end{scope}
\begin{scope}[xshift=13cm]
\node[draw, circle] (1) at (0,-1) {$1$};
\node[draw, circle] (2) at (2,0) {$2$};
\node[draw, circle] (3) at (2,-2) {$3$};
\node[draw, circle] (4) at (4,0) {$4$};
\node[draw, circle] (5) at (4,-2) {$5$};
\path[draw,thick,->] (1) -- (2);
\path[draw,thick,->] (1) -- (3);
\path[draw,thick,->] (3) -- (2);
\path[draw,thick,->] (2) -- (4);
\path[draw,thick,->] (3) -- (4);
\path[draw,thick,->] (5) -- (4);
\path[draw=red,thick,->,line width=2pt] (1) -- (3);
\path[draw=red,thick,->,line width=2pt] (3) -- (4);
\end{scope}
\end{tikzpicture}
\end{center}
\caption{Mahdolliset polut solmusta 1 solmuun 4.}
\label{fig:verpol}
\end{figure}

Polkujen määrän laskeminen on vaikea ongelma yleisessä
verkossa, jossa voi olla syklejä.
Itse asiassa tehtävä ei ole edes mielekäs sellaisenaan:
jos verkossa on sykli, voimme kiertää sykliä miten
monta kertaa tahansa ja tuottaa aina vain uusia polkuja,
joten polkuja tulee äärettömästi.
Nyt kuitenkin oletamme, että verkko on syklitön,
jolloin polkujen määrä on rajoitettu ja voimme laskea
sen tehokkaasti dynaamisella ohjelmoinnilla.

Jotta voimme käyttää dynaamista ohjelmointia,
meidän täytyy määritellä ongelma rekursiivisesti.
Sopiva funktio on $\texttt{polut}(x)$,
joka antaa polkujen määrän solmusta $a$ solmuun $x$.
Tätä funktiota käyttäen $\texttt{polut}(b)$
vastaa tehtävän ratkaisua.
Esimerkiksi kuvan \ref{fig:verpol} tilanteessa lähtösolmu on $a=1$
ja funktion arvot ovat seuraavat:

\begin{align*}
\texttt{polut}(1)&=1 \\
\texttt{polut}(2)&=2 \\
\texttt{polut}(3)&=1 \\
\texttt{polut}(4)&=3 \\
\texttt{polut}(5)&=0 \\
\end{align*}

Nyt meidän täytyy enää löytää tapa laskea
funktion arvoja.
Pohjatapauksessa olemme solmussa $a$,
jolloin on aina yksi tyhjä polku:

\[ \texttt{polut}(a)=1 \]

Entä sitten, kun olemme jossain muussa solmussa $x$?
Tällöin käymme läpi kaikki solmut, joista pääsemme
solmuun $x$ kaarella, ja laskemme yhteen näihin
solmuihin tulevien polkujen määrät.
Kun oletamme, että solmuun $x$ on kaari solmuista $u_1,u_2,\dots,u_k$,
saamme seuraavan rekursiivisen kaavan:
\[ \texttt{polut}(x)=\texttt{polut}(u_1)+\texttt{polut}(u_2)+\dots+\texttt{polut}(u_k) \]

Esimerkiksi kuvan \ref{fig:verpol} verkossa
solmuun $4$ on kaari solmuista $2$, $3$ ja $5$, joten
\[ \texttt{polut}(4)=\texttt{polut}(2)+\texttt{polut}(3)+\texttt{polut}(5) = 2+1+0 = 3.\]

Koska tiedämme, että verkko on syklitön,
voimme laskea funktion arvoja
tehokkaasti dynaamisella ohjelmoinnilla.
Oleellista on, että emme voi joutua koskaan silmukkaan
laskiessamme arvoja, ja
käytännössä laskemme arvot jossakin solmujen
topologisessa järjestyksessä.

\subsection{Ongelmat verkkoina}

Itse asiassa voimme esittää \emph{minkä tahansa}
dynaamisen ohjelmoinnin algoritmin
suunnatun syklittömän verkon käsittelynä.
Ideana on, että muodostamme verkon, jossa jokainen solmu on
yksi osaongelma ja kaaret ilmaisevat,
miten osaongelmat liittyvät toisiinsa.

Tarkastellaan esimerkkinä luvusta 9.1 tuttua tehtävää,
jossa haluamme laskea, monellako tavalla voimme muodostaa
korkeuden $n$ tornin, kun voimme käyttää palikoita,
joiden korkeudet ovat $1$, $2$ ja $3$.
Voimme esittää tämän tehtävän verkkona niin,
että solmut ovat tornien korkeuksia ja kaaret kertovat,
kuinka voimme rakentaa tornia palikoista.
Jokaisesta solmusta $x$ on kaari solmuihin
$x+1$, $x+2$ ja $x+3$,
ja polkujen määrä solmusta $0$ solmuun $n$
on yhtä suuri kuin tornin rakentamistapojen määrä.
Esimerkiksi kuva \ref{fig:verkol} näyttää verkon,
joka vastaa tapausta $n=4$.
Solmusta $0$ solmuun $4$ on yhteensä $7$ polkua,
eli voimme rakentaa korkeuden $4$ tornin $7$ tavalla.

\begin{figure}
\center
\begin{center}
\begin{tikzpicture}[scale=0.6]
\small
\node[draw, circle] (0) at (0,0) {$0$};
\node[draw, circle] (1) at (2,0) {$1$};
\node[draw, circle] (2) at (4,0) {$2$};
\node[draw, circle] (3) at (6,0) {$3$};
\node[draw, circle] (4) at (8,0) {$4$};
\path[draw,thick,->] (0) -- (1);
\path[draw,thick,->] (1) -- (2);
\path[draw,thick,->] (2) -- (3);
\path[draw,thick,->] (3) -- (4);
\path[draw,thick,->] (0) edge [bend left] (2);
\path[draw,thick,->] (1) edge [bend left] (3);
\path[draw,thick,->] (2) edge [bend left] (4);
\path[draw,thick,->] (0) edge [bend right] (3);
\path[draw,thick,->] (1) edge [bend right] (4);
\end{tikzpicture}
\end{center}
\caption{Tornitehtävän tapaus $n=4$ esitettynä verkkona.}
\label{fig:verkol}
\end{figure}

Olemme saaneet siis uuden tavan luonnehtia dynaamista ohjelmointia:
voimme käyttää dynaamista ohjelmointia,
jos pystymme esittämään ongelman suunnattuna syklittömänä verkkona.

\section{Vahvasti yhtenäisyys}

Jos suunnatussa verkossa on sykli,
emme voi muodostaa sille topologista järjestystä
emmekä käyttää dynaamista ohjelmointia.
Mikä neuvoksi, jos kuitenkin haluaisimme tehdä näin?

Joskus voimme selviytyä tilanteesta käsittelemällä
verkon vahvasti yhtenäisiä komponentteja.
Sanomme, että suunnattu verkko on \emph{vahvasti yhtenäinen},
jos mistä tahansa solmusta on polku mihin tahansa solmuun.
Voimme esittää suunnatun verkon aina yhtenä tai
useampana vahvasti yhtenäisenä komponenttina,
joista muodostuu syklitön \emph{komponenttiverkko}.
Tämä verkko esittää alkuperäisen verkon syvärakenteen.

\begin{figure}
\begin{center}
\begin{tikzpicture}[scale=0.6]
\small
\begin{scope}
\node[draw, circle] (1) at (0,0) {$1$};
\node[draw, circle] (2) at (0,-2) {$2$};
\node[draw, circle] (3) at (2,0) {$3$};
\node[draw, circle] (4) at (2,-2) {$4$};
\node[draw, circle] (5) at (4,0) {$5$};
\node[draw, circle] (6) at (4,-2) {$6$};

\path[draw,thick,->] (1) -- (3);
\path[draw,thick,->] (3) -- (4);
\path[draw,thick,->] (4) -- (2);
\path[draw,thick,->] (2) -- (1);
\path[draw,thick,->] (3) -- (2);

\path[draw,thick,->] (5) edge [bend left] (6);
\path[draw,thick,->] (6) edge [bend left] (5);

\path[draw,thick,->] (3) -- (5);
\path[draw,thick,->] (4) -- (6);

\draw[red,dashed,thick,line width=2pt] (-0.75,0.75) rectangle (2.75,-2.75);
\draw[red,dashed,thick,line width=2pt] (3.25,0.75) rectangle (4.75,-2.75);

\node at (2,-3.5) {(a)};
\end{scope}
\begin{scope}[xshift=8cm,yshift=-1cm]
\node[draw,rectangle] (1) at (0,0) {$\{1,2,3,4\}$};
\node[draw,rectangle] (2) at (4,0) {$\{5,6\}$};
\path[draw,thick,->] (1) -- (2);
\node at (2,-2.5) {(b)};
\end{scope}
\end{tikzpicture}
\end{center}
\caption{(a) Verkon vahvasti yhtenäiset komponentit.
(b) Komponenttiverkko, joka kuvaa verkon syvärakenteen.}
\label{fig:vahkom}
\end{figure}

Kuvassa \ref{fig:vahkom} on esimerkkinä verkko, joka muodostuu
kahdesta vahvasti yhtenäisestä komponentista.
Ensimmäinen komponentti on $\{1,2,3,4\}$
ja toinen komponentti on $\{5,6\}$.
Komponenteista muodostuu syklitön komponenttiverkko,
jossa on kaari solmusta $\{1,2,3,4\}$ solmuun $\{5,6\}$.
Tämä tarkoittaa, että voimme liikkua miten tahansa
joukon $\{1,2,3,4\}$ solmuissa sekä joukon $\{5,6\}$ solmuissa.
Lisäksi pääsemme joukosta $\{1,2,3,4\}$ joukkoon $\{5,6\}$,
mutta emme pääse takaisin joukosta $\{5,6\}$ joukkoon $\{1,2,3,4\}$.

\subsection{Kosarajun algoritmi}

Kosarajun algoritmi on tehokas algoritmi,
joka muodostaa suunnatun verkon vahvasti yhtenäiset
komponentit. Algoritmissa on kaksi vaihetta,
joista kumpikin käy läpi verkon solmut syvyyshaulla.
Ensimmäinen vaihe muistuttaa topologisen järjestyksen
etsimistä ja tuottaa listan solmuista.
Toinen vaihe muodostaa vahvasti yhtenäiset komponentit
tämän listan perusteella.

Algoritmin ensimmäisessä vaiheessa käymme läpi verkon
solmut ja aloitamme uuden syvyyshaun aina,
jos emme ole vielä käyneet solmussa.
Syvyyshaun aikana lisäämme solmun listalle,
kun olemme käyneet läpi kaikki solmusta lähtevät kaaret.
Toimimme siis kuten topologisen järjestyksen muodostamisessa,
mutta emme välitä, jos tulemme toista reittiä solmuun,
jota ei ole vielä käsitelty loppuun.

Algoritmin toisen vaiheen alussa
käännämme jokaisen verkon kaaren suunnan.
Tämän jälkeen käymme läpi käänteisessä järjestyksessä
listalla olevat solmut.
Aina kun vuoroon tulee solmu, jossa emme ole vielä käyneet,
aloitamme siitä solmusta syvyyshaun,
joka muodostaa uuden vahvasti yhtenäisen komponentin.
Lisäämme komponenttiin kaikki solmut,
joihin pääsemme syvyyshaun aikana ja jotka eivät
vielä kuulu mihinkään komponenttiin.

\begin{figure}
\center
\begin{center}
\begin{tikzpicture}[scale=0.6]
\scriptsize
\newcommand\verkko[7]{
\node[draw, circle, fill=#2] (1) at (0,0) {$1$};
\node[draw, circle, fill=#3] (2) at (0,-2) {$2$};
\node[draw, circle, fill=#4] (3) at (2,0) {$3$};
\node[draw, circle, fill=#5] (4) at (2,-2) {$4$};
\node[draw, circle, fill=#6] (5) at (4,0) {$5$};
\node[draw, circle, fill=#7] (6) at (4,-2) {$6$};

\path[draw,thick,->] (1) -- (3);
\path[draw,thick,->] (3) -- (4);
\path[draw,thick,->] (4) -- (2);
\path[draw,thick,->] (2) -- (1);
\path[draw,thick,->] (3) -- (2);
\path[draw,thick,->] (5) edge [bend left] (6);
\path[draw,thick,->] (6) edge [bend left] (5);
\path[draw,thick,->] (3) -- (5);
\path[draw,thick,->] (4) -- (6);

\node at (2,-3) {vaihe #1};
}
\begin{scope}
\verkko{1}{lightgray}{white}{white}{white}{white}{white}
\end{scope}
\begin{scope}[xshift=6cm]
\verkko{2}{lightgray}{white}{lightgray}{white}{white}{white}
\path[draw=red,thick,->,line width=2pt] (1) -- (3);
\end{scope}
\begin{scope}[xshift=12cm]
\verkko{3}{lightgray}{lightgray}{lightgray}{white}{white}{white}
\path[draw=red,thick,->,line width=2pt] (1) -- (3);
\path[draw=red,thick,->,line width=2pt] (3) -- (2);
\end{scope}
\begin{scope}[xshift=18cm]
\verkko{4}{lightgray}{lightgray}{lightgray}{white}{white}{white}
\path[draw=red,thick,->,line width=2pt] (1) -- (3);
\end{scope}
\begin{scope}[yshift=-5cm]
\verkko{5}{lightgray}{lightgray}{lightgray}{lightgray}{white}{white}
\path[draw=red,thick,->,line width=2pt] (1) -- (3);
\path[draw=red,thick,->,line width=2pt] (3) -- (4);
\end{scope}
\begin{scope}[yshift=-5cm,xshift=6cm]
\verkko{6}{lightgray}{lightgray}{lightgray}{lightgray}{white}{lightgray}
\path[draw=red,thick,->,line width=2pt] (1) -- (3);
\path[draw=red,thick,->,line width=2pt] (3) -- (4);
\path[draw=red,thick,->,line width=2pt] (4) -- (6);
\end{scope}
\begin{scope}[yshift=-5cm,xshift=12cm]
\verkko{7}{lightgray}{lightgray}{lightgray}{lightgray}{lightgray}{lightgray}
\path[draw=red,thick,->,line width=2pt] (1) -- (3);
\path[draw=red,thick,->,line width=2pt] (3) -- (4);
\path[draw=red,thick,->,line width=2pt] (4) -- (6);
\path[draw=red,thick,->,line width=2pt] (6) edge [bend left] (5);
\end{scope}
\begin{scope}[yshift=-5cm,xshift=18cm]
\verkko{8}{lightgray}{lightgray}{lightgray}{lightgray}{lightgray}{lightgray}
\path[draw=red,thick,->,line width=2pt] (1) -- (3);
\path[draw=red,thick,->,line width=2pt] (3) -- (4);
\path[draw=red,thick,->,line width=2pt] (4) -- (6);
\end{scope}
\begin{scope}[yshift=-10cm]
\verkko{9}{lightgray}{lightgray}{lightgray}{lightgray}{lightgray}{lightgray}
\path[draw=red,thick,->,line width=2pt] (1) -- (3);
\path[draw=red,thick,->,line width=2pt] (3) -- (4);
\end{scope}
\begin{scope}[yshift=-10cm,xshift=6cm]
\verkko{10}{lightgray}{lightgray}{lightgray}{lightgray}{lightgray}{lightgray}
\path[draw=red,thick,->,line width=2pt] (1) -- (3);
\end{scope}
\begin{scope}[yshift=-10cm,xshift=12cm]
\verkko{11}{lightgray}{lightgray}{lightgray}{lightgray}{lightgray}{lightgray}
\end{scope}
\end{tikzpicture}
\end{center}
\caption{Kosarajun algoritmin ensimmäinen vaihe.}
\label{fig:koseka}
\end{figure}

Tarkastellaan seuraavaksi, kuinka Kosarajun algoritmi
toimii esimerkkiverkossamme.
Kuva \ref{fig:koseka} näyttää algoritmin ensimmäisen vaiheen,
joka muodostaa solmuista listan $[2,5,6,4,3,1]$.
Kuva \ref{fig:kostok} näyttää algoritmin toisen vaiheen,
jossa käännämme ensin verkon kaaret ja
käymme sitten läpi solmut järjestyksessä $[1,3,4,6,5,2]$.
Vahvasti yhtenäiset komponentit syntyvät
solmuista 1 ja 6 alkaen.
Kaarten kääntämisen ansiosta
solmusta 1 alkava
vahvasti yhtenäinen komponentti ei ''vuoda''
solmujen 5 ja 6 alueelle.

\begin{figure}
\center
\begin{center}
\begin{tikzpicture}[scale=0.6]
\scriptsize
\newcommand\verkko[7]{
\node[draw, circle, fill=#2] (1) at (0,0) {$1$};
\node[draw, circle, fill=#3] (2) at (0,-2) {$2$};
\node[draw, circle, fill=#4] (3) at (2,0) {$3$};
\node[draw, circle, fill=#5] (4) at (2,-2) {$4$};
\node[draw, circle, fill=#6] (5) at (4,0) {$5$};
\node[draw, circle, fill=#7] (6) at (4,-2) {$6$};

\path[draw,thick,<-] (1) -- (3);
\path[draw,thick,<-] (3) -- (4);
\path[draw,thick,<-] (4) -- (2);
\path[draw,thick,<-] (2) -- (1);
\path[draw,thick,<-] (3) -- (2);
\path[draw,thick,<-] (5) edge [bend left] (6);
\path[draw,thick,<-] (6) edge [bend left] (5);
\path[draw,thick,<-] (3) -- (5);
\path[draw,thick,<-] (4) -- (6);

\node at (2,-3) {vaihe #1};
}
\begin{scope}
\verkko{1}{lightgray}{white}{white}{white}{white}{white}
\end{scope}
\begin{scope}[xshift=6cm]
\verkko{2}{lightgray}{lightgray}{white}{white}{white}{white}
\path[draw=red,thick,->,line width=2pt] (1) -- (2);
\end{scope}
\begin{scope}[xshift=12cm]
\verkko{3}{lightgray}{lightgray}{lightgray}{white}{white}{white}
\path[draw=red,thick,->,line width=2pt] (1) -- (2);
\path[draw=red,thick,->,line width=2pt] (2) -- (3);
\end{scope}
\begin{scope}[xshift=18cm]
\verkko{4}{lightgray}{lightgray}{lightgray}{white}{white}{white}
\path[draw=red,thick,->,line width=2pt] (1) -- (2);
\end{scope}
\begin{scope}[yshift=-5cm]
\verkko{5}{lightgray}{lightgray}{lightgray}{lightgray}{white}{white}
\path[draw=red,thick,->,line width=2pt] (1) -- (2);
\path[draw=red,thick,->,line width=2pt] (2) -- (4);
\end{scope}
\begin{scope}[yshift=-5cm,xshift=6cm]
\verkko{6}{lightgray}{lightgray}{lightgray}{lightgray}{white}{white}
\path[draw=red,thick,->,line width=2pt] (1) -- (2);
\end{scope}
\begin{scope}[yshift=-5cm,xshift=12cm]
\verkko{7}{lightgray}{lightgray}{lightgray}{lightgray}{white}{white}
\draw[red,dashed,thick,line width=1.5pt] (-0.75,0.75) rectangle (2.75,-2.75);
\end{scope}
\begin{scope}[yshift=-5cm,xshift=18cm]
\verkko{8}{lightgray}{lightgray}{lightgray}{lightgray}{white}{lightgray}
\end{scope}
\begin{scope}[yshift=-10cm]
\verkko{9}{lightgray}{lightgray}{lightgray}{lightgray}{lightgray}{lightgray}
\path[draw=red,thick,<-,line width=2pt] (5) edge [bend left] (6);
\end{scope}
\begin{scope}[yshift=-10cm,xshift=6cm]
\verkko{9}{lightgray}{lightgray}{lightgray}{lightgray}{lightgray}{lightgray}
\draw[red,dashed,thick,line width=1.5pt] (3.25,0.75) rectangle (4.75,-2.75);
\end{scope}
\end{tikzpicture}
\end{center}
\caption{Kosarajun algoritmin toinen vaihe.}
\label{fig:kostok}
\end{figure}

Keskeinen kysymys Kosarajun algoritmiin liittyen on,
miten voimme olla varmoja, että algoritmin toinen vaihe
muodostaa vahvasti yhtenäisiä komponentteja,
joihin ei tule ylimääräisiä solmuja.

Voimme tarkastella asiaa muodostettavan
komponenttiverkon näkökul\-masta.
Jos meillä on komponentti $A$, josta pääsee kaarella
komponenttiin $B$, 
algoritmin ensimmäisessä vaiheessa
jokin $A$:n solmu lisätään listalle kaikkien $B$:n
solmujen jälkeen.
Kun sitten käymme läpi listan käänteisessä järjestyk\-sessä,
jokin $A$:n solmu tulee vastaan ennen kaikkia $B$:n
solmuja.
Niinpä alamme rakentaa ensin komponenttia $A$
emmekä mene komponentin $B$ puolelle,
koska verkon kaaret on käännetty.
Sitten kun myöhemmin muodostamme komponentin $B$,
emme mene käännettyä kaarta komponenttiin $A$,
koska komponentti $A$ on jo muodostettu.

\subsection{Esimerkki: Luolapeli}

Olemme pelissä luolastossa, joka muodostuu $n$ luolasta ja
joukosta käytäviä niiden välillä.
Jokainen käytävä on yksisuuntainen.
Jokaisessa luolassa on yksi aarre, jonka voimme ottaa mukaamme,
jos kuljemme luolan kautta.
Peli alkaa luolasta $1$ ja päättyy luolaan $n$.
Montako aarretta voimme saada, jos valitsemme parhaan
mahdollisen reitin?
On sallittua kulkea saman luolan kautta monta kertaa tarvittaessa.

\begin{figure}
\center
\begin{center}
\begin{tikzpicture}[scale=0.6]
\small
\node[draw, circle] (1) at (0,0) {$1$};
\node[draw, circle] (2) at (2,-1) {$2$};
\node[draw, circle] (3) at (-2,0) {$3$};
\node[draw, circle] (4) at (0,-2) {$4$};
\node[draw, circle] (5) at (0,-4) {$5$};
\node[draw, circle] (6) at (-2,-2) {$6$};
\node[draw, circle] (7) at (-2,-4) {$7$};

\path[draw,thick,->] (1) -- (4);
\path[draw,thick,->] (4) -- (2);
\path[draw,thick,->] (2) -- (1);
\path[draw,thick,->] (4) -- (6);
\path[draw,thick,->] (1) -- (3);
\path[draw,thick,->] (3) -- (6);
\path[draw,thick,->] (6) edge [bend left] (7);
\path[draw,thick,->] (7) edge [bend left] (6);
\path[draw,thick,->] (4) -- (5);
\path[draw,thick,->] (5) -- (7);
\end{tikzpicture}
\end{center}
\caption{Luolasto, jossa on 7 luolaa ja 10 käytävää.
Haluamme kulkea luolasta 1 luolaan 7 keräten mahdollisimman paljon aarteita.}
\label{fig:luopel}
\end{figure}

Voimme mallintaa tilanteen verkkona, jonka solmut ovat luolia ja
kaaret ovat käytäviä. Haluamme löytää reitin solmusta $1$ solmuun $n$
niin, että kuljemme mahdollisimman monen solmun kautta.
Esimerkiksi kuva \ref{fig:luopel} näyttää verkkona luolaston, joka muodostuu
seitsemästä luolasta ja kymmenestä käytävästä.
Yksi optimaalinen reitti luolasta $1$ luolaan $7$ on
$1 \rightarrow 4 \rightarrow 2 \rightarrow 1 \rightarrow 3 \rightarrow 6 \rightarrow 7$,
jota seuraten saamme kerättyä kaikki aarteet paitsi luolassa 5 olevan aarteen.
Ei ole olemassa reittiä, jota noudattamalla saisimme haltuumme kaikki luolaston aarteet.

\begin{figure}
\center
\begin{center}
\begin{tikzpicture}[scale=0.6]
\small
\node[draw, rectangle] (1) at (0,0) {$\{1,2,4\}$};
\node[draw, rectangle] (2) at (-4,-2) {$\{3\}$};
\node[draw, rectangle] (3) at (4,-2) {$\{5\}$};
\node[draw, rectangle] (4) at (0,-4) {$\{6,7\}$};

\path[draw,thick,->] (1) -- (2);
\path[draw,thick,->] (1) -- (3);
\path[draw,thick,->] (1) -- (4);
\path[draw,thick,->] (2) -- (4);
\path[draw,thick,->] (3) -- (4);
\end{tikzpicture}
\end{center}
\caption{Luolaston vahvasti yhtenäiset komponentit.}
\label{fig:luovah}
\end{figure}

Voimme ratkaista ongelman tehokkaasti määrittämällä ensin verkon
vahvasti yhtenäiset komponentit.
Tämän jälkeen riittää löytää sellainen polku alkusolmun komponentista
loppusolmun komponenttiin, että komponenttien kokojen summa
on suurin mahdollinen.
Koska verkko on syklitön, tämä onnistuu dynaamisella ohjelmoinnilla.

Kuva \ref{fig:luovah} näyttää vahvasti yhtenäiset komponentit
esimerkkiverkossamme.
Tästä esityksestä näemme suoraan, että optimaalisia reittejä on olennaisesti kaksi:
voimme kulkea joko luolan 3 tai luolan 5 kautta.