\chapter{Aikajana}

Tietojenkäsittelytiede alkoi eriytyä omaksi tieteenalakseen
1950-luvulla, ja monet tässä kirjassa esitellyt tekniikat
löydettiin alan ensimmäisinä vuosikymmeninä.
Seuraava lista kokoaa yhteen joitakin kirjan aiheita
julkaisuvuoden mukaan.
Usein on kuitenkin mahdotonta määrittää,
kuka todellisuudessa keksi jonkin menetelmän ensimmäisenä.

\subsection*{1945}

\begin{itemize}
\item
Lomitusjärjestäminen (von Neumann)
\end{itemize}

\subsection*{1956}

\begin{itemize}
\item
Kruskalin algoritmi
\end{itemize}

\subsection*{1957}

\begin{itemize}
\item
Primin algoritmi
\end{itemize}

\subsection*{1958}

\begin{itemize}
\item
Bellman–Fordin algoritmi
\end{itemize}

\subsection*{1959}

\begin{itemize}
\item
Dijkstran algoritmi
\item
Pikajärjestäminen (Hoare)
\end{itemize}

\subsection*{1962}

\begin{itemize}
\item
AVL-puu (Adelson-Velsky ja Landis)
\item
Floyd–Warshallin algoritmi
\item
Ford–Fulkersonin algoritmi
\end{itemize}

\subsection*{1964}

\begin{itemize}
\item
Kekojärjestäminen (Williams)
\end{itemize}

\subsection*{1971}

\begin{itemize}
\item
P vs. NP (Cook)
\end{itemize}

\subsection*{1972}

\begin{itemize}
\item
Edmonds–Karpin algoritmi
\end{itemize}

\subsection*{1978}

\begin{itemize}
\item
Kosarajun algoritmi
\end{itemize}
