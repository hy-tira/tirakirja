\chapter{Johdanto}

\section{Laskennalliset ongelmat}

\section{Rekursiiviset algoritmit}

\subsection{Osajoukkojen läpikäynti}

\subsection{Permutaatioiden läpikäynti}

\subsection{Peruuttava haku}

\section{Matemaattinen tausta}

\subsection{Summakaavat}

Lukujen $1,2,\dots,n$ summaa vastaa kaava
\[1+2+\dots+n = \frac{n(n+1)}{2}.\]
Esimerkiksi
\[1+2+3+4+5 = \frac{5 \cdot 6}{2}=15.\]
Kaavan voi ymmärtää niin, että laskemme yhteen $n$ lukua,
joiden suuruus on \emph{keskimäärin} $(n+1)/2$.

Toinen hyödyllinen kaava on
\[2^0+2^1+\dots+2^n = 2^{n+1}-1.\]
Esimerkiksi
\[1+2+4+8+16=32-1.\]
Tämän kaavan voi ajatella niin, että aloitamme luvusta $2^n$
ja lisäämme siihen aina puolet pienemmän luvun lukuun $1$ asti.
Niinpä pääsemme yhtä vaille lukuun $2^{n+1}$ asti.

\subsection{Logaritmit}

Logaritmin määritelmän mukaan $\log_b(n)=a$
tarkalleen silloin kun $b^a=n$.
Esimerkiksi $\log_2(32)=5$, koska $2^5=32$.

Algoritmiikassa logaritmin kantaluku $b$ on usein 2,
ja voimme ajatella, että logaritmi kertoo, montako kertaa
meidän tulee puolittaa luku $n$, ennen kuin pääsemme lukuun 1.
Esimerkiksi $\log_2(32)=5$, koska tarvitsemme 5 puolitusta:
\[32 \rightarrow 16 \rightarrow 8 \rightarrow 4 \rightarrow 2 \rightarrow 1\]

Logaritmeille pätee kaavat
\[\log_b(x \cdot y) = \log_b(x)+\log_b(y)\]
ja
\[\log_b(x / y) = \log_b(x)-\log_b(y).\]
Ylemmästä kaavasta seuraa myös
\[\log_b(x^k) = k \log_b(x).\]
Lisäksi voimme vaihtaa logaritmin kantalukua kaavalla
\[\log_u(x) = \frac{\log_b(x)}{\log_b(u)}.\]