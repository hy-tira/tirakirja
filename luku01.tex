\chapter{Johdanto}

\section{Laskennalliset ongelmat}

\section{Rekursiiviset algoritmit}

\subsection{Osajoukkojen läpikäynti}

Joukon osajoukkoja ovat kaikki tavat valita osa joukon alkioista.
Esimerkiksi joukon $\{1,2,3\}$ osajoukot ovat
$\emptyset$ (tyhjä joukko), $\{1\}$, $\{2\}$, $\{3\}$,
$\{1,2\}$, $\{1,3\}$, $\{2,3\}$ ja $\{1,2,3\}$.
Jos joukossa on $n$ alkioita, osajoukkoja on $2^n$.

Rekursio tarjoaa kätevän tavan käydä läpi kaikki
joukon osajoukot. Esimerkiksi seuraava koodi pitää yllä
rakennetta

\begin{code}
ArrayDeque<Integer> osajoukko;
\end{code}

joka sisältää vuorollaan kunkin joukon $\{1,2,\dots,n\}$
osajoukon. Rekursiivista metodia \texttt{muodosta} kutsutaan
parametrilla $1$.

\begin{code}
void muodosta(int x) {
    if (x == n+1) {
        System.out.println(osajoukko);
        return;
    }
    muodosta(x+1); // x ei valita osajoukkoon
    osajoukko.addLast(x);
    muodosta(x+1); // x valitaan osajoukkoon
    osajoukko.removeLast(x);
}
\end{code}

Jokaisessa kutsussa metodi käy läpi tapaukset,
otetaanko luku $x$ mukaan osajoukkoon vai ei.
Molemmissa tapauksissa metodi kutsuu itseään yhtä
suuremalla $x$:n arvolla.
Lopulta kun $x=n+1$, kaikki luvut on käyty läpi
ja on aika tulostaa osajoukko.

Esimerkiksi tapauksessa $n=3$ koodin tulostus on seuraava:

\begin{code}
[]
[3]
[2]
[2, 3]
[1]
[1, 3]
[1, 2]
[1, 2, 3]
\end{code}

\subsection{Permutaatioiden läpikäynti}

Joukon permutaatiot ovat kaikki tavat järjestää joukon alkiot.
Esimerkiksi joukon $\{1,2,3\}$ permutaatiot ovat
$(1,2,3)$, $(1,3,2)$, $(2,1,3)$, $(2,3,1)$, $(3,1,2)$ ja $(3,2,1)$.
Jos joukossa on $n$ alkiota, permutaatioita on $n!$.

Myös permutaatioiden läpikäynti onnistuu kätevästi rekursiolla.
Seuraava koodi pitää yllä rakennetta

\begin{code}
ArrayDeque<Integer> permutaatio;
\end{code}

joka sisältää vuorollaan kunkin joukon $\{1,2,\dots,n\}$ permutaation.
Rekursiivista metodia \texttt{muodosta} kutsutaan ilman parametreja.

\begin{code}
void muodosta() {
    if (permutaatio.size() == n) {
        System.out.println(permutaatio);
        return;
    }
    for (int i = 1; i <= n; i++) {
        if (!permutaatio.contains(i)) {
            permutaatio.addLast(i);
            muodosta();
            permutaatio.removeLast();
        }
    }
}
\end{code}

Tässä on ideana, että metodi käy joka kutsulla läpi kaikki luvut
$1,2,\dots,n$ ja aina jos luku ei kuulu vielä permutaatioon,
koodi haarautuu rekursiivisesti tapaukseen, jossa se lisätään
seuraavaksi permutaatioon.
Sitten kun permutaatiossa on $n$ lukua, se on valmis ja
voimme tulostaa sen.

Esimerkiksi tapauksessa $n=3$ koodin tulostus on seuraava:

\begin{code}
[1, 2, 3]
[1, 3, 2]
[2, 1, 3]
[2, 3, 1]
[3, 1, 2]
[3, 2, 1]
\end{code}

\subsection{Peruuttava haku}

Peruuttava haku on yleinen rekursiivinen menetelmä,
joka muodostaa järjes\-telmällisesti kaikki ratkaisut tehtävään.
Siinä on ideana aloittaa tyhjästä ratkaisusta ja käydä
joka askeleella läpi kaikki mahdolliset tavat laajentaa ratkaisua.

\section{Matemaattinen tausta}

\subsection{Summakaavat}

Lukujen $1,2,\dots,n$ summaa vastaa kaava
\[1+2+\dots+n = \frac{n(n+1)}{2}.\]
Esimerkiksi
\[1+2+3+4+5 = \frac{5 \cdot 6}{2}=15.\]
Kaavan voi ymmärtää niin, että laskemme yhteen $n$ lukua,
joiden suuruus on \emph{keskimäärin} $(n+1)/2$.

Toinen hyödyllinen kaava on
\[2^0+2^1+\dots+2^n = 2^{n+1}-1.\]
Esimerkiksi
\[1+2+4+8+16=32-1.\]
Tämän kaavan voi ajatella niin, että aloitamme luvusta $2^n$
ja lisäämme siihen aina puolet pienemmän luvun lukuun $1$ asti.
Niinpä pääsemme yhtä vaille lukuun $2^{n+1}$ asti.

\subsection{Logaritmit}

Logaritmin määritelmän mukaan $\log_b(n)=a$
tarkalleen silloin kun $b^a=n$.
Esimerkiksi $\log_2(32)=5$, koska $2^5=32$.

Algoritmiikassa logaritmin kantaluku $b$ on usein 2,
ja voimme ajatella, että logaritmi kertoo, montako kertaa
meidän tulee puolittaa luku $n$, ennen kuin pääsemme lukuun 1.
Esimerkiksi $\log_2(32)=5$, koska tarvitsemme 5 puolitusta:
\[32 \rightarrow 16 \rightarrow 8 \rightarrow 4 \rightarrow 2 \rightarrow 1\]

Logaritmeille pätee kaavat
\[\log_b(x \cdot y) = \log_b(x)+\log_b(y)\]
ja
\[\log_b(x / y) = \log_b(x)-\log_b(y).\]
Ylemmästä kaavasta seuraa myös
\[\log_b(x^k) = k \log_b(x).\]
Lisäksi voimme vaihtaa logaritmin kantalukua kaavalla
\[\log_u(x) = \frac{\log_b(x)}{\log_b(u)}.\]