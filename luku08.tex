\chapter{Dynaaminen ohjelmointi}

Dynaaminen ohjelmointi on algoritmisuunnittelun tekniikka,
jota voi käyttää kahdenlaisissa tilanteissa:

\begin{itemize}
\item \textbf{Optimiratkaisun etsiminen}: Haluamme etsiä ratkaisun,
joka on jollain tavalla suurin tai pienin.
\item \textbf{Ratkaisumäärän laskeminen}: Haluamme laskea,
montako erilaista ratkaisua on olemassa.
\end{itemize}

Ideana on muotoilla laskentatehtävä
rekursiivisesti niin, että voimme ratkaista tehtävän
ratkaisemalla ensin osatehtävinä saman tehtävän pienempiä tapauksia.
Aina kun olemme ratkaisseet tietyn osatehtävän,
kirjaamme sen ratkaisun muistiin, minkä ansiosta
pystymme hakemaan ratkaisun tehokkaasti uudelleen,
jos tarvitsemme sitä myöhemmin.

Tässä luvussa tutustumme ensin dynaamisen ohjelmoinnin perusteisiin
käyttäen esimerkkinä tehtävää, jossa haluamme jakaa annetun
rahamäärän kolikoiksi.
Tämän jälkeen käymme läpi joukon muita tehtäviä, jotka esittelevät
dynaamisen ohjelmoinnin mahdollisuuksia.

\section{Perustekniikat}

Ensimmäinen esimerkkitehtävämme on seuraava:
Meillä on joukko kolikoita sekä rahamäärä,
jonka haluamme jakaa kolikoiksi.
Voimme käyttää jokaista kolikkoa miten tahansa
monta kertaa ratkaisussa.
Esimerkiksi jos kolikot ovat $\{1,3,4\}$
ja rahamäärä on $5$, yksi mahdollinen
jakotapa on $1+1+3=5$.
Haluamme selvittää vastauksen kahteen kysymykseen:

\begin{itemize}
\item \textbf{Optimiratkaisun etsiminen}:
Mikä on pienin määrä kolikoita, joilla voimme muodostaa rahamäärän?
\item \textbf{Ratkaisumäärän laskeminen}:
Monellako eri tavalla voimme muodostaa rahamäärän kolikoista?
\end{itemize}

Yllä olevassa esimerkissä pienin määrä kolikoita on 2,
koska voimme tehdä jaon $1+4=5$. Erilaisia ratkaisuja taas on 6:

\begin{multicols}{2}
\begin{itemize}
\item $1+1+1+1+1=5$
\item $1+1+3=5$
\item $1+3+1=5$
\item $3+1+1=5$
\item $1+4=5$
\item $4+1=5$
\end{itemize}
\end{multicols}

Osoittautuu, että molemmat kysymykset ratkeavat tehokkaasti
dynaamisella ohjelmoinnilla
etsimällä niille sopiva rekursiivinen esitys.

\subsection{Optimiratkaisun etsiminen}

Ensimmäinen tehtävämme on selvittää, mikä on pienin määrä
kolikoita, joilla voimme muodostaa annetun rahamäärän.
Tätä varten määrittelemme funktion $\texttt{pienin}(x)$,
joka antaa pienimmän määrän kolikoita, joilla voimme
muodostaa rahamäärän $x$.
Esimerkiksi kun kolikot ovat $\{1,3,4\}$,
niin $\texttt{pienin}(5)=2$, mikä vastaa ratkaisua $1+4=5$.

Jotta voimme käyttää dynaamista ohjelmointia,
meidän täytyy pystyä laskemaan funktion \texttt{pienin}
arvo rekursiivisesti kutsumalla sitä itseään.
Pohjatapauksena on $\texttt{pienin}(0)=0$, koska voimme
muodostaa rahamäärän 0 käyttämättä yhtään kolikkoa.
Lisäksi on hyödyllistä määritellä $\texttt{pienin}(x)=\infty$,
kun $x<0$. Tämä tarkoittaa, että ei ole sallittua muodostaa
ratkaisua, jossa rahamäärä olisi ääretön.

Kuinka voimme sitten laskea arvon $\texttt{pienin}(x)$,
missä $x>0$? Nyt on kätevää tarkastella \emph{viimeistä}
kolikkoa, joka kuuluu ratkaisuun.
Esimerkiksi jos kolikot ovat $\{1,3,4\}$, viimeinen kolikko
on joko 1, 3 tai 4.
Jos viimeinen kolikko on 1, meidän tulee muodostaa vielä
rahamäärä $x-1$, mihin kuluu $\texttt{pienin}(x-1)$ kolikkoa.
Vastaavasti jos viimeinen kolikko on 3 tai 4,
meidän tulee muodostaa rahamäärä $x-3$ tai $x-4$.
Niinpä saamme seuraavan rekursiivisen kaavan tapaukseen $x>0$:
\[
\texttt{pienin}(x) =
    \min(\texttt{pienin}(x-1),\texttt{pienin}(x-3),\texttt{pienin}(x-4))
\]
Tässä tapauksessa funktion \texttt{pienin} ensimmäiset arvot ovat:

\begin{center}
\begin{tabular}{rrrrrrrrrrrr}
$x$ & 0 & 1 & 2 & 3 & 4 & 5 & 6 & 7 & 8 & 9 & 10 \\
\hline
$\texttt{pienin}(x)$ & 0 & 1 & 2 & 1 & 1 & 2 & 2 & 2 & 2 & 3 & 3 \\
\end{tabular}
\end{center}


Tarkastellaan sitten yleistä tilannetta, jossa meillä on $k$ kolikkoa,
joiden arvot ovat $\{c_0,c_1,\ldots,c_{k-1}\}$.
Nyt voimme jälleen muodostaa rekursiivisen kaavan samalla idealla:
\[
\texttt{pienin}(x) =
    \min(\texttt{pienin}(x-c_0),\texttt{pienin}(x-c_1),\dots,\texttt{pienin}(x-c_{k-1}))
\]



\subsection{Ratkaisumäärän laskeminen}

\section{Esimerkkejä}

\subsection{Pisin nouseva alijono}

\subsection{Repunpakkaus}

\subsection{Reitti ruudukossa}
