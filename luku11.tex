\chapter{Reitinhaku}

Reitinhaku on keskeinen algoritmiikan ongelma, jossa haluamme
löytää parhaan reitin paikasta $a$ paikkaan $b$.
Esimerkiksi voimme haluta etsiä nopeimman tavan
matkustaa kotoa yliopistolle
tai halvimman lentoyhteyden kaupungista $a$
kaupunkiin $b$.
Verkkojen kielellä sanomme, että tavoitteemme on löytää \emph{lyhin polku}
solmusta $a$ solmuun $b$.

Jos verkon kaarilla ei ole painoja,
voimme löytää lyhimmät polut helposti leveyshaun avulla,
kuten olemme tehneet edellisessä luvussa.
Tässä luvussa kuitenkin keskitymme tilanteeseen,
jossa verkon kaarilla on painot.
Tässä tilanteessa tarvitsemme kehittyneempiä algoritmeja
lyhimpien polkujen löytämiseen.

Tutustumme kolmeen keskeiseen reitinhakumenetelmään:

\begin{itemize}
\item \textbf{Bellman–Fordin algoritmi} etsii lyhimmät polut
lähtösolmusta kaikkiin muihin solmuihin ajassa $O(nm)$.
\item \textbf{Dijkstran algoritmi} etsii lyhimmät polut
lähtösolmusta kaikkiin muihin solmuihin \emph{tehokkaammin} ajassa $O(m \log n)$
olettaen, että verkossa ei ole negatiivisen painoisia kaaria.
\item \textbf{Floyd–Warshallin algoritmi} etsii lyhimmät polut
\emph{kaikkien} verkon solmuparien välillä ajassa $O(n^3)$.
\end{itemize}

\section{Bellman–Fordin algoritmi}

Bellman–Fordin algoritmi laskee jokaiselle verkon solmulle,
kuinka pitkä on lyhin polku lähtösolmusta $x$ solmuun.
Algoritmi pitää yllä verkon solmuille arvioita,
mikä on etäisyys solmusta $x$ kyseiseen solmuun.
Aluksi etäisyys lähtösolmuun on 0 ja etäisyys kaikkiin muihin
solmuihin on $\infty$ (ääretön).
Sitten algoritmi alkaa parantaa etäisyyksiä
hyödyntämällä verkon kaaria.

Algoritmi muodostuu $n-1$ kierroksesta, joista jokainen käy läpi
kaikki verkon kaaret.
Jokaisen kaaren $a \rightarrow b$ kohdalla algoritmi tarkastaa,
voiko kaaren avulla parantaa etäisyyttä solmusta $x$ solmuun $b$
kulkemalla solmun $a$ kautta.
Jos näin on, algoritmi merkitsee uuden etäisyyden muistiin.
Algoritmin päätteeksi jokaisen solmun etäisyys on sama kuin
lyhimmän polun pituus solmusta $x$ kyseiseen solmuun.

Kuva X näyttää esimerkin Bellman–Fordin algoritmin toiminnasta.
Tässä lähtösolmuna on solmu 1, ja haluamme selvittää siitä
lyhimmät polut muihin solmuihin.
Jokaisen solmun vieressä on ilmoitettu sen etäisyysarvio. TODO

\subsection{Analyysi}

Bellman–Fordin algoritmi muodostuu $n-1$ kierroksesta,
joista jokainen käy läpi verkon $m$ kaarta,
joten on helppoa nähdä, että algoritmi toimii ajassa $O(nm)$.
Mutta miksi on varmaa, että $n-1$ kierroksen jälkeen olemme
määrittäneet kaikki lyhimmät polut?

Ensimmäinen havainto on,
että $n$ solmun verkossa jokainen lyhin polku voi
sisältää enintään $n-1$ kaarta.
Jos polkuun kuuluisi $n$ tai enemmän kaaria,
jokin solmu esiintyisi polulla monta kertaa.
Tämä ei ole kuitenkaan mahdollista,
koska ei olisi järkeä kulkea monta kertaa saman solmun kautta,
kun haluamme saada aikaan lyhimmän polun.

Toinen havainto on, että jos lyhin polku solmusta $x$ solmuun $y$ on
$x \rightarrow s_1 \rightarrow s_2 \rightarrow \dots \rightarrow s_k \rightarrow y$,
niin myös lyhin polku solmusta $x$ solmuun $s_1$ on $x \rightarrow s_1$,
lyhin polku solmusta $x$ solmuun $s_2$ on $x \rightarrow s_1 \rightarrow s_2$, jne.,
eli jokainen polun alkuosa on myös lyhin polku vastaavaan solmuun.
Jos näin ei olisi, voisimme parantaa lyhintä polkua solmusta $x$ solmuun $y$
parantamalla jotain polun alkuosaa.

Tarkastellaan nyt, mitä tapahtuu algoritmin kierroksissa.
Ensimmäisen kierroksen jälkeen olemme löytäneet lyhimmät polut,
joissa on yksi kaari.
Toisen kierroksen jälkeen olemme löytäneet lyhimmät polut,
joissa on kaksi kaarta.
Sama jatkuu, kunnes $n-1$ kierroksen jälkeen olemme löytäneet
lyhimmät polut, joissa on $n-1$ kaarta.
Koska missään lyhimmässä polussa ei voi olla enempää kaaria,
olemme löytäneet kaikki lyhimmät polut.

Huomaa, että Bellman–Fordin algoritmi ei toimi järkevästi,
jos verkossa on \emph{negatiivinen sykli}
eli sykli, jonka kokonaispituus on negatiivinen.
Esimerkki tästä on kuvan X verkossa oleva sykli X.
Tällöin voimme lyhentää polkua loputtomasti kulkemalla
sykliä uudestaan ja uudestaan, eikä lyhimmän polun pituudella ole alarajaa.
Voimme havaita negatiivisen syklin olemassaolon suorittamalla
algoritmia $n$ kierrosta tavallisen $n-1$ sijaan.
Jos viimeisellä kierroksella jokin etäisyys pienenee,
verkossa on negatiivien sykli.

\subsection{Toteutus}

Bellman–Fordin algoritmi on kätevää toteuttaa käyttäen verkon
kaarilista\-esitystä. Meillä on siis lista

\begin{code}
ArrayList<Kaari> kaaret = new ArrayList<>();
\end{code}

jonka jokainen alkio vastaa yhtä kaarta.
Lisäksi määrittelemme taulukon

\begin{code}
int[] etaisyys = new int[n+1];
\end{code}

joka pitää kirjaa solmujen etäisyyksistä.
Alustamme taulukon näin:

\begin{code}
for (int i = 1; i <= n; i++) {
    etaisyys[i] = 1e9;
}
etaisyys[x] = 0;
\end{code}

Koska Javassa ei ole käsitettä ''ääretön'', käytämme sen sijasta
suurta lukua $10^9$, eli oletamme, että kaikki etäisyydet
ovat todellisuudessa pienempiä kuin $10^9$.
Jos etäisyydet voivat olla suurempia, tätä lukua täytyy muuttaa.

Tämän jälkeen voimme toteuttaa algoritmin näin:

\begin{code}
for (int i = 1; i <= n-1; i++) {
    for (Kaari kaari : kaaret) {
        int vanha = etaisyys[kaari.mihin];
        int uusi = etaisyys[kaari.mista]+kaari.paino;
        etaisyys[kaari.mihin] = Math.min(vanha, uusi);
    }
}
\end{code}

\section{Dijsktran algoritmi}

\section{Floyd-Warshallin algoritmi}
