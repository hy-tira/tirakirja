\chapter{Algoritmien suunnittelu}

Kuinka voi suunnitella tehokkaan algoritmin?
On selvää, ettei tähän kysymykseen ole helppoa vastausta.
Yhtä hyvin voisi kysyä, kuinka voi kirjoittaa hyvän romaanin
tai säveltää hyvää musiikkia.

Tällä kurssilla yleinen tavoitteemme on
saada aikaan tehokas algoritmi, joka toimisi ajassa $O(n)$ tai $O(n \log n)$.
Kun tämä tavoite on tiedossa, voimme ottaa sen algoritmin
suunnittelun lähtökohdaksi ja rajata sen avulla mahdollisia
lähestymistapoja, joita voimme käyttää.

\section{Tehokkaat algoritmit}

Millainen on algoritmi, joka vie aikaa $O(n)$ tai $O(n \log n)$?
Yksi hyvä kuvaus on,
että kun algoritmille annetaan $n$ alkion aineisto,
se saa käyttää jokaisen alkion käsittelyyn
vain pienen määrän aikaa.
Tämä tarkoittaa käytännössä, että algoritmissa saa
esiintyä seuraavan kaltaisia silmukoita:

\begin{code}
for (int i = 0; i < n; i++) {
    // tee jotain nopeaa
}
\end{code}

Tässä ''jotain nopeaa'' tarkoittaa koodia, joka vie aikaa
$O(1)$ tai $O(\log n)$.
Lisäksi koska järjestäminen vie aikaa $O(n \log n)$,
algoritmi voi halutessaan järjestää aineistoa.
Kovin paljon muuta tehokas algoritmi ei sitten voikaan tehdä
aineistolle.

Tämä rajoittaa paljon mahdollisuuksiamme suunnitella algoritmia,
mutta voimme ajatella asiaa myös myönteisesti:
vaatimus tehokkuudesta rajaa pois suuren määrän lähestymistapoja,
eli meidän on helpompaa löytää hyvä algoritmi,
kun vaihtoehtojen määrä on pienempi.

\section{Esimerkkejä}

\subsection{X}

\subsection{X}

\subsection{X}
